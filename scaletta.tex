\section{Scaletta}

\begin{itemize}

  \item[\textcolor{red}{1}] \textcolor{red}{Short bodies $1 \le \AR \le 2$: Effect of $\AR$ on modes $A$, $B$, and $QP$}
  \begin{itemize}
    \item[\textcolor{red}{1.1}] How does $Re_{c,2}$ changes with $\AR$? What about $\beta_c$? How does the sequence of bifurcation change? Do new modes arise in this range of $\AR$? Floquet stability analysis for $\AR=1,1.25,1.5,1.75$ and $Re=150,175,200,225$ (We need somo more $Re$). Show the $Re-\beta$ curves. Show the dependence of $\mu_{max}$ and $\beta_{\max}$ on $\AR$ and $Re$. Characterise the new modes that arise, showing the eigenmode and their symmetry. Compare with literature; see Choi \& Yang (JFM,2014).
    \item[\textcolor{red}{1.2}] For intermediate $\AR$s (see $\AR=1.25$) the base-flow topology changes at larger $Re$, and also the successive flow bifurcations. Characterise the base-flow dependence on $Re$, e.g. show the dependence of $T$ and on the features of the mean flow on $\AR$. Show the results of the 2D and 3D Floquet analysis at different $Re$. Characterise the new modes. Look at different $\AR$s.
  \end{itemize}
  
  \item[\textcolor{blue}{2}] \textcolor{blue}{Intermediate $\AR \approx 3$: A far wake instability}
  \begin{itemize}
    \item[\textcolor{blue}{2.1}] Use $\AR=3$ and $Re=450$ as an example. Show that the wake can be divided into three part, (i) near wake with the classical vortex shedding, (ii) middle wake where the positive/negative vortex monopoles split, (iii) far wake with a low-frequency oscillation. Characterise the three parts using $u-v$ diagrams at different $(x,0)$, highlight the frequency at different positions. Use different codes to validate.
    \item[\textcolor{blue}{2.2}] How does this scenario change with $\AR$ and $Re$? Does it hold also for different geometries? What about the circular cylinder case? Is it possible that the far wake bifurcation actually arises also for classical wake once the other flow bifurcations are stabilised?
    \item[\textcolor{blue}{2.3}] Can we perform a local stability analysis? What about stability analysis of the vertical velocity profile under the assumption of quasi parallel flow? What about Floquet analysis of the truncated domain (we interpolate the base flow on a grid that contains only the periodic part)?
  \end{itemize}
  
  \item[\textcolor{Fuchsia}{3}] \textcolor{Fuchsia}{Intermediate $ 4 \le \AR \le 4.9$: A slanted wake}
  \begin{itemize}
    \item[\textcolor{Fuchsia}{3.1}] Use $\AR=4.5$ as an example. Show the results from the 3D DNS. At lower $Re$ the flow is $2D$ and the wake is straight, with the classical alternation of positive/negative monopoles of vorticity. At larger $Re$ the wake becomes suddenly $3D$ and slanted. Use $u-v$  and/or $C_\ell-C_d$ plots in phase space and frequency spectra.
    \item[\textcolor{Fuchsia}{3.2}] Floquet analysis for the 2D flow bifurcation. Compare with Jallas et al. (2017). Show the multipliers and the unstable modes. Explain the physics looking at the interaction between monopoles of same/different sign. Subcritical bifurcation, look at the normal form, at the non linear evolution of the mode, report the value of the Landau constant.
    \item[\textcolor{Fuchsia}{3.3}] Floquet analysis for the 3D flow bifurcation. Show the multipliers and the mode. Highlight the large increase of the growth rate with $Re$. The bifurcation is of subharmonic nature. Same results have been observed also for $\AR=4$.
  \end{itemize}
  
  \item[\textcolor{olive}{4}] \textcolor{olive}{Elongated bodies $\AR \ge 5$: What rules, the TE or LE vortex shedding?}
  \begin{itemize}
    \item[\textcolor{olive}{4.1}] Depending on $\AR$ the flow dynamics depends on either the LE or TE vortex shedding. This influences as well the secondary flow bifurcation. Show the Floquet multipliers and the mode for $\AR=5.25,5.5,5.75,6$. In the oblique branch mode $A$ appears again. In this case the flow is first unstable to 3D perturbations and the wake drives the flow instability. For longer bodies the instability is instead driven by mode $QS$. Show the dependence of the multipliers on $Re$ and the structural sensitivity. 
    \item[\textcolor{olive}{4.2}] Provide more details on the physics looking at the budget for the energy and the enstrophy. Can we say something more?
    \item[\textcolor{olive}{4.3}] Non linear simulations
  \end{itemize}
  
\end{itemize}
    
     
