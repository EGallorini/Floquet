\subsection{Sensitivity to a variation of $Re$}

We now investigate the sensitivity map of the Floquet exponent to a variation of the Reynolds number, i.e. $\partial \sigma/\partial Re$.
%
We put ourself at $Re \approx Re_c$, and consider a small depatrture from criticality; we define a small $\epsilon$ such that
%
\begin{equation}
  \frac{1}{Re_c} - \frac{1}{Re} = \epsilon.
\end{equation}
%
This small departure leads to a small variation of the base flow, of the eigenvector and of the eigenvalue, i.e.
%
\begin{equation}
  \{ \bm{u}_0, p_0 \} \rightarrow \{ \bm{u}_0, p_0 \} + \{\delta \bm{u}_0,\delta p_0\}, \ \{ \bm{u}_1, p_1 \} \rightarrow \{ \bm{u}_1, p_1 \}  + \{\delta \bm{u}_1,\delta p_1\}\ \text{and} \ \sigma \rightarrow \sigma + \delta \sigma.
\end{equation}
%
In addition to this, the way the flow moves along the limit cycles changes. Following \cite{}, we introduce a phase coordinate $\tau$ along the limit cycle, and expand $\text{d}\tau/\text{d}t$ in $\epsilon$ such that
%
\begin{equation}
  \frac{\text{d} \tau}{\text{d} t} = 1 + \epsilon \dot{\tau}_1 +  \mathcal{O}(\epsilon^2).
\end{equation}
%
Thus we now write $\{\bm{u}_i(\tau(t)),p_i(\tau(t))\}$ and using the chain rule the time derivative of the velocity becomes
%
\begin{equation}
  \frac{\partial \bm{u}_i}{\partial t} = \frac{\partial}{\partial \tau} \frac{\text{d} \tau}{\text{d} t} = ( 1 + \epsilon \dot{\tau}_1 + \mathcal{O}(\epsilon^2)) \frac{\partial \partial \bm{u}_i}{\partial t}.
\end{equation} 
%
We now introduce these modifications in the non linear equations for the base flow $\{\bm{u}_0,p_0\}$ and in the linearised equations for the perturbation $\{\bm{u}_1,p_1\}$.
%
We start from the equation for the base flow. We replace $\{\bm{u}_0,p_0\}$ with $\{\bm{u}_0,p_0\} + \{\delta \bm{u}_0,\delta p_0\}$ and consider a small departure from $Re_c$. By recalling that equation for $\{\bm{u}_0,p_0\}$ at $Re=Re_c$ holds, and considering only first order terms we obtain
%
\begin{equation}
  \begin{aligned}
  \underbrace{\begin{pmatrix} \mathcal{I} & 0 \\ 0^T & 0 \end{pmatrix}}_B
  \frac{\partial}{\partial \tau} \underbrace{\begin{pmatrix} \delta \bm{u}_0 \\ \delta p_0 \end{pmatrix}}_{\delta \bm{q}_0} +
  \underbrace{ \begin{pmatrix} \mathcal{C}( \bm{u}_0, \cdot ) + \mathcal{C}(\cdot,\bm{u}_0)-\frac{1}{Re_c} \bm{\nabla}^2(\cdot) & \bm{\nabla}(\cdot) \\
                     \bm{\nabla} \cdot ( \cdot )                                                                   &  0   \end{pmatrix}}_{\mathcal{L}(\bm{u}_0,Re_c)}
                     \begin{pmatrix} \delta \bm{u}_0 \\ \delta p_0 \end{pmatrix} = \\
                     - \epsilon \underbrace{ \begin{pmatrix} \bm{\nabla}^2 \bm{u}_0 \\ 0 \end{pmatrix} }_{\mathcal{F}_2^0}- 
                       \epsilon \dot{\tau}_1 \frac{\partial}{\partial \tau}  \begin{pmatrix} \mathcal{I} & 0 \\ 0^T & 0 \end{pmatrix}\underbrace{ \begin{pmatrix} \bm{u}_0 \\  p_0 \end{pmatrix} }_{\bm{q}_0}
  \end{aligned}
\end{equation}
%
where $\mathcal{C}(\bm{u}_A,\bm{u}_B) = \left( \bm{u}_A \cdot \bm{\nabla} \right) \bm{u}_B$. This equation can be compactly written as 
%
\begin{equation}
  B \frac{\partial \delta \bm{q}_0}{\partial \tau} - \mathcal{L}(\bm{u}_0,Re_c) \delta \bm{q}_0 = - \epsilon \mathcal{F}_2^0 - \epsilon \dot{\tau}_1 B \frac{\partial \bm{q}_0}{\partial \tau}.
\end{equation}
%
The operator has a zero eigenvalue with algebraic multiplicity two. Specifically, at the critical Reynolds number $Re=Re_c$, there are two Floquet multipliers $\mu=(1,0)$ located on the unit circle. One corresponds to the ''trivial'' Floquet exponent, associated with the eigenvector $\hat{\bm{q}}_0 = \partial \bm{q}_0/\partial \tau$ \citep{}, while the other is associated with the actual flow bifurcation, represented by the eigenmode $\hat{\bm{q}}_1 = \{ \hat{\bm{u}}_1, \hat{p}_1 \}$. To solve this problem, we must impose a solvability condition, which requires that the component of the forcing term on the right-hand side along the kernel of the operator vanishes. This condition is enforced by exploiting the degree of freedom in $\dot{\tau}_1$. This condition is automatically satisfied for the eigenvalue corresponding to the synchronous symmetry-breaking bifurcation, as the right-hand side is orthogonal to the associated eigenvector $\hat{\bm{q}}_1$ by symmetry. The solvability condition thus requires that
%
\begin{equation}
  \langle \bm{\bm{q}}_0^\dagger, - \mathcal{F}_2^0 - \dot{\tau}_1 B \hat{\bm{q}}_0 \rangle = 0,
\end{equation} 
%
which provides the expression of $\dot{\tau}_1$, i.e.
%
\begin{equation}
  \dot{\tau}_1 = - \frac{ \langle \hat{\bm{q}}_0^\dagger, \mathcal{F}_2^0 \rangle }{ \langle \hat{\bm{q}}_0^\dagger, B \hat{\bm{q}}_0 \rangle}
\end{equation}
%
where $\hat{\bm{q}}_0^\dagger$ indicates the adjoint mode and $\langle \bm{a},\bm{b} \rangle = \int_{t_0}^{t_0+T} \int_{\Omega} \left( \bm{a}^H \bm{b} \right) \text{d}\Omega \text{d}t$ the inner product, with $\cdot^H$ being denoting the transconjugate.
%
We then solve for $\delta \bm{q}_0$ as
%
\begin{equation}
  \begin{cases}
  B \frac{\partial \delta \bm{q}_0}{\partial \tau} - \mathcal{L}(\bm{u}_0,Re_c) \delta \bm{q}_0 = - \epsilon \mathcal{F}_2^0 - \epsilon \dot{\tau}_1 B \frac{\partial \bm{q}_0}{\partial \tau} \\
  \langle \hat{\bm{q}}_0^\dagger, B \delta \bm{q}_0 \rangle = 0
  \end{cases}.
\end{equation}
%
The second additional condition ensures uniqueness by selecting a single solution from the infinite solution set. At this point we solve for $\bm{q}_0^\epsilon  = \delta \bm{q}_0/\epsilon$. To accelerate the convergence of $\bm{q}_0^\epsilon$ toward a periodic solution, we employ a Newton method coupled with a GMRES algorithm built upon Arnoldi iteration \citep{}.

We now move to the equation for the perturbation $\bm{q} = \{ \bm{u}_1 p_1 \}$. We replace $\bm{q}_0$ with $\bm{q}_0 + \delta \bm{q}_0$ and $\bm{q}_1$ with $\bm{q}_1 + \delta \bm{q}_1$. Using the Floquet ansatz we write:
%
\begin{equation}
  \begin{gathered}
  \frac{\partial \bm{u}_1}{\partial t} = \frac{\partial}{\partial t} \left( \hat{\bm{u}}_1(\tau) e^{\sigma \tau} \right) = 
  \frac{\partial \hat{\bm{u}}_1}{\partial \tau} \frac{\text{d} \tau}{\text{d} t} e^{\sigma \tau} +
  \hat{\bm{u}}_1 \sigma \frac{\text{d} \tau}{\text{d} t} e^{\sigma \tau} = \\
  = \frac{\partial \hat{\bm{u}}_1}{\partial \tau} \left( 1 + \epsilon \dot{\tau}_1 +  \mathcal{O}(\epsilon^2) \right) e^{\sigma \tau} +
  \hat{\bm{u}}_1 \sigma ( 1 + \epsilon \dot{\tau}_1 + \mathcal{O}(\epsilon^2) ) e^{\sigma \tau} = \\
 = \left( \frac{\partial \hat{\bm{u}}_1}{\partial \tau} + \frac{\partial \hat{\bm{u}}_1}{\partial \tau} \epsilon \dot{\tau}_1 + \hat{\bm{u}}_1 \sigma + \hat{\bm{u}}_1 \sigma \epsilon \dot{\tau}_1 \right) e^{\sigma \tau} + \mathcal{O}(\epsilon^2).
  \end{gathered}
\end{equation}
Following a similar procedure as for the base flow modification and keeping only first order terms we obtain
%
\begin{equation}
  \begin{gathered}
  B \frac{\partial \delta \hat{\bm{q}}_1}{\partial \tau} +
  \epsilon \dot{\tau}_1 B \frac{ \partial \hat{\bm{q}}_1 }{\partial \tau} +
  \delta \sigma B \hat{\bm{q}}_1 +
  \sigma B \delta \hat{\bm{q}}_1 + 
  \sigma \epsilon \dot{\tau}_1 B \hat{\bm{q}}_1 +
  \mathcal{L}(\bm{u}_0,Re_c) \delta \hat{\bm{q}}_1 = -\epsilon \underbrace{ \left( \mathcal{F}_{a} + \mathcal{F}_{b} \right) }_{\mathcal{F}},
  \end{gathered}
\end{equation}
%
where
%
\begin{equation}
  \mathcal{F}_{a} = \begin{pmatrix} \bm{\nabla}^2 \hat{u}_1 \\ 0 \end{pmatrix}
  \qquad \text{and} \qquad
  \mathcal{F}_{b} = \begin{pmatrix} \mathcal{C}(\hat{\bm{u}}_1,\bm{u}_0^\epsilon) + \mathcal{C}(\bm{u}_0^\epsilon,\hat{\bm{u}}_1) \\ 0 \end{pmatrix}.
\end{equation}

We now isolate $\delta \sigma$. To remove $\delta \hat{\bm{q}}_1$ we project this equation along the adoint direction $\hat{\bm{q}}_1^\dagger$ to otbain
%
\begin{equation}
  \begin{gathered}
  \delta \sigma \langle \hat{\bm{q}}_1^\dagger, B \hat{\bm{q}}_1 \rangle +
  \epsilon \langle \hat{\bm{q}}_1^\dagger, \dot{\tau}_1 B \frac{\partial \hat{\bm{q}} }{\partial \tau} \rangle +
  \epsilon \langle \hat{\bm{q}}_1^\dagger, \sigma \dot{\tau}_1 B \hat{\bm{q}}_1 \rangle =  
  - \epsilon \langle \hat{\bm{q}}_1^\dagger, \mathcal{F} \rangle.
  \end{gathered}
\end{equation}
%
Therefore, we do have
%
\begin{equation}
  \delta \sigma = - \epsilon \frac{ \langle \hat{\bm{q}}_1^\dagger, \mathcal{F}  + \mathcal{G} \rangle }
                                  { \langle \hat{\bm{q}}_1^\dagger, B \hat{\bm{q}}_1 \rangle }
\end{equation}
%
where 
%
\begin{equation}
  \mathcal{G} = \begin{pmatrix} \dot{\tau}_1 \frac{\partial \hat{\bm{u}}_1}{\partial \tau} + \sigma \dot{\tau}_1 \hat{\bm{u}}_1 \\ 0 \end{pmatrix}.
\end{equation}
%
At this point we write
%
\begin{equation}
  \frac{\partial \sigma}{\partial Re} = 
  \frac{\partial \sigma}{\partial \epsilon} \frac{\partial \epsilon}{\partial Re} = 
  \frac{1}{Re^2} \frac{\partial \sigma}{\partial \epsilon} = 
  - \frac{1}{Re} \left( \frac{ \langle \hat{\bm{q}}_1^\dagger, \mathcal{F}  + \mathcal{G} \rangle }
                                  { \langle \hat{\bm{q}}_1^\dagger, B \hat{\bm{q}}_1 \rangle } \right)
\end{equation}
%
The idea is therefore to plots at different phases the integrand funciton inside the inner porduct at the numerator, to highlight which region and which term is mostly responsible for a variation of the global eigenvalues after a small variaiton of the Reynolds number.
